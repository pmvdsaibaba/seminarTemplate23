%!TEX root=../main.tex
\section{Modular Design of Secure Group Messaging Protocols and the Security of MLS}
\label{sec:CCS:ACDT21}
(Pothula, Saibaba, \cite{CCS:ACDT21,EPRINT:ACDT21})\\

Messaging Layer Security (MLS) tries to standardize a protocol for Secure Group Messaging (SGM). The authors of this publication focused on the problem of addressing the complexity and formally capturing the security of SGM protocols, presenting MLS as an SGM protocol with modular components, and providing instantiations for these components.

\subsection{Motivation}

MLS has evolved over time. It seems like the few important sub protocols like Continuous Group Key agreement (CGKA) have been prominently the focus in SGM of MLS. Unlike that the motivation here lies in advancing the understanding of MLS and SGM protocols as a whole by looking into their complexity and contributing to defining and proving security for such protocols.

\subsection{Main Idea}

Main contribution from the paper is a security proof showing the basic instantiation of the MLS as an SGM. They gave a black box construction of SGM protocols from a CGKA protocol and other primitives namely, Forward Secure Group AEAD (FS-GAEAD) and PRF-PRNG. They gave a proof that the safety predicate of the SGM construction is secure as long as the underlying scheme’s predicates are secure. The complexity of the SGM security definition arises from the bookkeeping of states. They developed a paradigm of History graph to reduce this complexity and give a visualization of stored data and also define the security of the protocols.\\

MLS requires an accurate modeling of PKI. Addressing limitations in previous works and unlike earlier models that made simplifying assumptions, they proposed a PKI model. They implemented PKI with a mechanism to authenticate each other's long-term keys through an out-of-band mechanism while distributing short term public keys via an untrusted key server. Also they proved RTeeKEM, which uses updatable public key encryption, is a passively secure CGKA.

\subsection{Relation to Other Papers}

It is mentioned that the history graph paradigm was introduced in the paper and later used in other works. As mentioned in this paper, Formal Models and Varied Protocols for Group Messaging: Attacks and Proofs for IETF MLS, detailed formal specifications and mechanized security proofs for multiple protocols in MLS (up to draft 7).


\subsection{Criteria for Comparison}

For similar constructions of the complete MLS protocols, maybe the adversarial  capabilities could be compared. 
